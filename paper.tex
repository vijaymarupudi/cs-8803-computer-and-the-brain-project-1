\documentclass{article}

\title{CS 8803: Computation and the Brain - Project 1}

\begin{document}
\maketitle

\textbf{An algorithms that is sensitive to geometric properties of parallelism and convexity}

\subsection{Parallel lines}

\begin{itemize}
\item run canny filter, but with an additional step. We store continuous runs of edges in a stack
\item then we use a heuristic to detect whether it is a straight line. This can be a projection onto a linear plane $y = mx + b$. We have a threshold for the error term that is configurable.
\item Then we use the fitted linear line to determine the angle compared to the x axis.
\item we then generate a rotational matrix that rotates the image so that the targeted line is flat.
\item We then create a convolution matrix that corresponds to the line we want to find parallel lines to.
\item We then run this convolution over the image.
\item The closer to the sum of all the numbers in the convolutional matrix a specific point is, the higher the probability of there being a parallel line.

\end{itemize}

\subsection{Topology}

\begin{itemize}
\item use an object detection algorithm to determine the objects in the image.
\item pick an object to focus on, and then crop the image to only have that object.
\item run the canny filter, so that only the edge is visible.
\item run the graham scan algorithm
\end{itemize}


\end{document}
