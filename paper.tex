\documentclass[11pt]{article}

\usepackage{amsmath}
\usepackage[margin=1in]{geometry}
\title{CS 8803: Computation and the Brain Project 1: A parallelism detector}

\author{Vijay Marupudi, Taylor Del Matto, Abrahim Laddha, \& Bianca Smart}
\date{March 1st, 2021}

\begin{document}

\maketitle

\section{Algorithm}

For the sake of simplicity, we assume that the input image can be represented
as a matrix of numbers $A$ of dimensions $m$ by $n$ where $A_{ij} \in [0, 1]$. This algorithm leaves the possibility of being extended to color images.

The edge detection algorithm is a modified version of the Canny edge detector, which can be implemented as a series of convolutions and image filters that are neurally plausible.

We first assume that biological eyes do not perfect resolution of the image
they are seeing, so we apply a Gaussian convolutional filter over the image to
get output matrix $B$. This also helps get rid of noise effects on the model.

$$B = \begin{bmatrix}
  0.25 & 0.5 &  0.25 \\
  0.5 & 1 & 0.5 \\
  0.25 & 0.5 & 0.25
\end{bmatrix} * A$$

We can then use Sobel filter convolution to find edges in the horizontal (high values of $H$ matrix) and vertical (high values of $V$ matrix) directions. Pixels in the matrix that are not part of an edge will have a close to 0 value as a result of these filters.

for e.g.

\[
H = \begin{bmatrix}
  1 & 2 & 1 \\
  0 & 0 & 0 \\
  -1 & -2 & -1
\end{bmatrix} * B
\]

Then we can find the hypotenuse between $H$ and $V$ in order to determine the value of the edge as a specific point in matrix $S$ and the direction in matrix $D$.

\begin{align*}
  S &= \sqrt{H^2 + V^2} \\
  D &= \text{atan2}(H, V)
\end{align*}

Now we can use matrix $S$ and $D$ to find continuous running lines in the image. We first round matrix $D$ to angles of 0, 45, 90, 135 so that it is easy to calculate what the next pixel in a running edge is.

Further edge thinking techniques can be applied using $S$ and $D$, but will not be mentioned to stay within the page limit.

We will now maintain a list of continuous edges in the image. We start at a
pixel in $S$, and following the running edge until it ends, and store the
information about the edge run in a list. For $S_{ij}$, we can refer to
$D_{ij}$ for the direction of the edge. For example, if $D_{ij}$ has been
rounded to 45 degrees, then the continuing edge is probably at $S_{i+1, j + 1}$
and $S_{i-1, j-1}$. We then check if the values of $S_{i+1, j+1}$ and $S_{i-1,
j-1}$ are greater than learnable threshold parameter $\lambda$. If not, we do
not need to further explore in that direction. If it is greater, then we
continue down the edge until it is not.

Once an edge is found, we can use a technique similar to linear regression to
determine whether it is straight, using the x and y values of the pixels to
find the slope $m$, akin to $mx + b$. This can be done using projection, which
will also provide use with the error vector $\epsilon$.  Learnable parameter
$\alpha$ can determine the amount of acceptable error. If $\sum
\frac{\epsilon^2}{y^2} > \alpha$, then the algorithm will move on the next
edge, as it is not straight enough. This requirement for the edge to be
``straight enough'' can be relaxed to make this algorithm adopt other geometric properties like symmetry.

If the error is lower than $\alpha$, then the matrix that describes the edge
acts like a convolution matrix on matrix $S$, to determine other appropriate
parallel lines in the image. This edge matrix will be padded with negative
number on either side to ensure that it is sensitive to other edges and not
react positively to just white parts of the image. For example, for specific
edge matrix $E$, we pad it with negative numbers to make $E_\text{convolution}$
and then convolve it with the edges of the whole image in $S$ to get matrix $P$, with information about parallel lines.

$$E = \begin{bmatrix}
  1 & 0 & 0 \\
  0 & 1 & 0 \\
  0 & 0 & 1
\end{bmatrix}$$


$$E_\text{convolution} = \begin{bmatrix}
  -1 & -1 & 0 & 0 & 0\\
  -1 & 1  & -1 & 0 & 0 \\
  0  & -1 & 1 & -1 & 0 \\
  0  & 0  & -1 & 1 & -1 \\
  0  & 0  & 0 & -1 & -1 \\
\end{bmatrix}$$

$$P = E_\text{convolution} * S$$

Relatively high values in $P$ represent locations in the image that are parallel to the edge.

We can now repeat this algorithm for each edge.

% \subsection{Topology}

% \begin{itemize}
% \item use an object detection algorithm to determine the objects in the image.
% \item pick an object to focus on, and then crop the image to only have that object.
% \item run the canny filter, so that only the edge is visible.
% \item run the graham scan algorithm
% \end{itemize}

\end{document}
